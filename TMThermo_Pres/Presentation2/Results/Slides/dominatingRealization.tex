\begin{frame}{Dominating Realization}
\begin{block}{Heat Function for Dominating Realization of TM $M$}
Can be shown that $G(x) = \ln(2)K[x|\phi_M(x)]$ satisfies condition 2 of Prop 1. Thus
\begin{equation*}
    Q_\text{dom} = kT\ln(2)K[x|\phi_M(x)]
\end{equation*}
is the heat function for a realization, called the \textit{dominating realization}, of TM $M$.
\begin{itemize}
    \item Inputs generating a lot of heat are large and incompressible, and $\phi_M$ is non-invertible for that input
    \item Inputs generating little heat are those for which $\phi_M$ is invertible
    \begin{itemize}
        \item For these inputs, $Q(x) = O(1)$
    \end{itemize}
\end{itemize}
\end{block}
\end{frame}

%Computability Constraints
\begin{frame}{Dominating Realization}
\begin{block}{Non-Computability}
\begin{equation*}
    Q_\text{dom} = kT\ln (2)K[x|\phi_M(x)]
\end{equation*}
    \begin{itemize}
        \item Dominating realization is not computable
        \item It is upper semi-computable
        \begin{itemize}
            \item Can be obtained in limit by sequence of increasingly efficient computable realizations $Q_n(x)$
            \item Converges on $Q_\text{dom}(x)$ from above
        \end{itemize}
    \end{itemize}
    \end{block}
\end{frame}

%Optimality of Dominating Realization
\begin{frame}{Dominating Realization}
\begin{block}{Efficiency of Dominating Realization}
    For any other \textit{computable} realization with heat function $Q(X)$:
    \begin{equation*}
        Q(x) \ge Q_\text{dom} - kT\left[\ln(2)K(Q/kT) + K(\phi_M)\right] + O(1)
    \end{equation*}
    \begin{itemize}
        \item $Q_\text{dom}$ is minimal up to a negative constant. 
        \begin{itemize}
            \item For $Q(x) \le Q_\text{dom}$, $\phi_M$ has to have high complexity, or $Q$ has to have high complexity
        \end{itemize}
        \item The above inequality only holds for computable realizations. 
    \end{itemize}
\end{block}
\end{frame}

%Applicability
\begin{frame}{Dominating Realization}
\begin{block}{Applicability of the Dominating Realization}
	\begin{align*}
        Q(x) \ge Q_\text{dom} - kT\left[\ln(2)K(Q/kT) + K(\phi_M)\right] + O(1)
    \end{align*}
    \begin{itemize}
    	\item Above inequality only holds if LHS is $Q$ for a \textit{computable} realization
    	\item Validity of Church-Turing Thesis extends applicability to all physical realizations
    \end{itemize}
\end{block}
\end{frame}

%Comparison with Coin-Flipping Realization
\begin{frame}{Dominating Realization}
\begin{block}{Comparison With Coin-Flipping Realization}
\begin{itemize}
	\item Coin-Flipping Realization uncomputable, thus, not necessarily true that $Q\sub{coin}(x) \ge Q\sub{dom}(x) + O(1)$. Can be shown that
	\begin{equation*}
	Q\sub{coin}(x) \ge Q\sub{dom}(x) + O\{\log(K[\phi_U(x)])\}
	\end{equation*}
	\item Minimal heat production for output $y$:
	\begin{itemize}
		\item $Q\sub{coin}$ minimized for $\ell(x) = K(\phi_M(x))$. Finding such a program is uncomputable.
		\item $Q\sub{dom}$ minimized if $\phi_M^{-1}(y) = x$. Finding such a program as simple as ``print $y$".
	\end{itemize}
	\item Coin-Flipping realization has 0 EP on coin-flipping distribution. Dominating realization has $EP > 0 $ for all input distributions
\end{itemize}
\end{block}
\end{frame}

%Introduce Heat vs Complexity Trade-off
\begin{frame}{Dominating Realization}
\begin{block}{Heat VS. Complexity Trade-off}
     \begin{equation*}
        Q(x) \ge Q_\text{dom} - kT[\ln 2K(Q/kT) + K(\phi_M)] + O(1)
    \end{equation*}
    Using $Q_\text{dom} = kT\ln 2K[x|\phi_M(x)]$ and re-arraigning gives:
    \begin{equation*}
        Q(x)/\ln 2 + K(Q) + K(f) \ge K(x|y) + O(1)
    \end{equation*}
    \begin{itemize}
        \item Every computation mapping $x$ to $y$ comes with a "cost`` of $K(x|y)$
        \item Cost can be paid by generating heat, having a high complexity heat function, or having a high complexity mapping $f$
    \end{itemize}
\end{block}
\end{frame}
