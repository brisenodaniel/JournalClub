% re-iterate importance of Physical Church-Turing Thesis
\begin{frame}{Physical Church-Turing Thesis}
\begin{block}{Significance of Physical Church Turing Thesis}
\begin{itemize}
    \item Current conclusions only apply to computable realizations
    \item In principle, non-computable realizations of TM could exist 
    \item Validity of Church-Turing Thesis would imply any physical realization of a TM must follow thermodynamic constraints shown in paper
\end{itemize}
\end{block}
\end{frame}

%closing remarks
\begin{frame}{Conclusion}
\begin{block}{Final Remarks}
\begin{itemize}
    \item Proposition 1 allows us to relate logical properties of a TM to its thermodynamic properties.
    \item Coin-flipping realization gives a highly thermodynamically reversible case
    \begin{itemize}
        \item Infinite expected heat for zero EP input distribution
        \item Heat minimizing input distribution implies nonzero EP
    \end{itemize}
    \item Dominating realization gives lower bound on heat production for any computable realization
    \begin{itemize}
        \item Upper semicomputable
        \item The inequality $ Q(x)/\ln 2 + K(Q) + K(f) \ge K(x|y) + O(1)$ allows us to decompose intrinsic cost of mapping $x\mapsto y$ into complexity of heat function, complexity of mapping, and heat production.
    \end{itemize}
\end{itemize}
\end{block}
\end{frame}